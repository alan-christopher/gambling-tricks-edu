\PassOptionsToPackage{table}{xcolor}
\documentclass[pdf]{beamer}
\mode<presentation>{\usetheme{Dresden}}
\usepackage{lmodern}
\usepackage[normalem]{ulem}
\usepackage{amsmath,textcomp,amssymb,geometry,graphicx,listings,array,color,amsthm}

\usepackage[noend]{algpseudocode}

\usepackage[
backend=biber,
sorting=ynt,
]{biblatex}
\addbibresource{lecture.bib}

\usepackage{tikz}
\usepackage{multicol}
\usetikzlibrary{shapes,snakes}
\usetikzlibrary{positioning}
\usetikzlibrary{arrows}
\usetikzlibrary{fit}
\usetikzlibrary{math}

%% For on-slide alerting of nodes
\tikzstyle{alert} = [text=black, fill=blue!20, draw=black]
\setbeamercolor{alerted text}{fg=blue}
\tikzset{alerton/.code args={<#1>}{%
  \only<#1>{\pgfkeysalso{alert}} % \pgfkeysalso doesn't change the path
}}

%% utils for removing uninteresting sections from the navbar
%% https://tex.stackexchange.com/questions/317774/hide-section-from-sidebar
\makeatletter
\let\beamer@writeslidentry@miniframeson=\beamer@writeslidentry%
\def\beamer@writeslidentry@miniframesoff{%
  \expandafter\beamer@ifempty\expandafter{\beamer@framestartpage}{}% does not happen normally
  {%else
    % removed \addtocontents commands
    \clearpage\beamer@notesactions%
  }
}
\newcommand*{\miniframeson}{\let\beamer@writeslidentry=\beamer@writeslidentry@miniframeson}
\newcommand*{\miniframesoff}{\let\beamer@writeslidentry=\beamer@writeslidentry@miniframesoff}
\makeatother

%% preamble
\title{Beating the House}
\subtitle{You will not beat the house}
\author{A.C.}
\date{\today}

\AtBeginSection[]
{
  \miniframesoff
  \begin{frame}{Outline}
    \tableofcontents[currentsection,hideothersubsections]
  \end{frame}
  \miniframeson
}

\definecolor{darkred}{rgb}{0.7,0,0}
\definecolor{darkgreen}{rgb}{0,0.5,0}
\definecolor{darkblue}{rgb}{0,0,0.5}
\definecolor{darkpurple}{rgb}{0.4, 0.0, 0.4}

%% Code font settings
\lstset{
  showstringspaces=false,
  basicstyle=\scriptsize\ttfamily,
  commentstyle=\color{darkred},
  stringstyle=\color{darkgreen},
  keywordstyle=\bfseries\color{darkpurple},
}

%%%%%%%%%%%%%%%%%%%%%%%%%%
% Start of Actual slides %
%%%%%%%%%%%%%%%%%%%%%%%%%%
\begin{document}
\begin{frame}
  \titlepage
\end{frame}

\section{Putting It All On Black}

\subsection{Playing it Straight}

\begin{frame}{The Rules}
  \begin{enumerate}
  \item Place a bet with probability $p$ and payout ratio $r$.
  \item The wheel spins.
  \item Get paid \ldots or don't.
    \begin{enumerate}
      \item House advantage: $pr < 1$
    \end{enumerate}
  \end{enumerate}

  \[ X = \alpha \left(-1 + r\mathbb{I}(\text{Win})\right) \]
\end{frame}

\begin{frame}{Open Decisions}
  \begin{itemize}
  \item \sout{What bet to place.}
    \begin{itemize}
    \item Black feels lucky.
    \item $p \approx 0.47$
    \item $r = 2$
    \end{itemize}
  \item How much to bet.
  \item When to walk away.
  \end{itemize}
\end{frame}

\subsection{Playing it Straight}
\begin{frame}{The System}
  \begin{algorithmic}
    \Procedure{ConstantBets}{}
    \State $\alpha \gets 1$
    \While{\text{true}}
      \State $\text{bet}(\alpha)$
    \EndWhile
    \EndProcedure
  \end{algorithmic}
\end{frame}

\begin{frame}{Playing it Out}
  \[ W_n \sim \text{Bin(n, p)} \]
  \[ X_n = r \cdot W_n - n\]

  \pause
  \[ \text{Var}(X_n) = r^2 \text{Var}(W_n) = npqr^2\]
  \[ E(X_n) = r\cdot E(W_n) - n = npr - n = n(pr -1) \]
  \pause

  Expected losses grow linearly in $n$, stddev as $\sqrt{n}$. Per Chebyshev, we
are going broke, \emph{almost certainly}.
\end{frame}

\subsection{Double or Nothing!}
\begin{frame}{The System}
  \begin{algorithmic}
    \Procedure{MartingaleBetting}{}
    \State $\alpha \gets 1$
    \While{\text{true}}
      \State $\text{bet}(\alpha)$
      \If{$\text{Win}$}
        \State \Return
      \Else
        \State $\alpha \gets 2 \cdot \alpha$
      \EndIf
    \EndWhile
    \EndProcedure
  \end{algorithmic}
\end{frame}

\begin{frame}{Playing it Out}
  Let $N$ be the round of betting where we finally win:

  \[ X_N = -1 - 2 - 4 \cdots -2^{N-1} + 2^N = 1 \]

  $p$ nowhere to be found. Guaranteed profit!

  \pause
  On \emph{arbitrarily} bad bets!
\end{frame}

\begin{frame}{Alas!}
  \begin{itemize}
  \item Very hard to get unlimited betting rounds.
    \begin{itemize}
    \item Credit limits
    \item Table limits
    \end{itemize}
  \end{itemize}

  \pause
  Exercise for the reader: what happens when we get cut off after $N$ rounds?
\end{frame}

\subsection{Never Say Die}
\begin{frame}{The System}
  \begin{algorithmic}
    \Procedure{AntiMartingaleBetting}{}
    \State $\text{reserves} \gets 1.0$
    \State $\text{target} \gets 1.0$
    \While{$\text{reserves} \leq \text{target}$}
      \State $\alpha \gets \text{reserves} \cdot 0.5$
      \State $\text{bet}(\alpha)$
      \If{$\text{Win}$}
        \State $\text{reserves} \gets \text{reserves} + \alpha$
      \Else
        \State $\text{reserves} \gets \text{reserves} - \alpha$
      \EndIf
    \EndWhile
    \EndProcedure
  \end{algorithmic}
\end{frame}

\begin{frame}{Playing it Out}

  \[ P(\text{Lose}) = 0 \]

  \pause

  \[ P_{-1} = 1 \]
  \[ P_{0} = p + q P_{1}\]
  \[ P_{1} = p P_{0}  + q P_{2} \]
  \[ P_{n} = p P_{n-1} + qP_{n+1} \]
  \[ pP_{n} + qP_n = p P_{n-1} + qP_{n+1} \]

  \pause
  Massage. Massage.
  
  \pause
  % TODO: verify this by hand. As it is this is by sloppy reduction to https://www.columbia.edu/~ks20/FE-Notes/4700-07-Notes-GR.pdf
  \[ P_0 = \frac{p}{q} \]
  
\end{frame}

\begin{frame}{Alas!}
  We've created a gambler's ruin problem, just flipped and in log scale.

  \begin{itemize}
  \item Nonzero probability of winning \emph{at every step}.
  \item Nonzero probability of never walking away from the table,
  \item Also, need to be able to place arbitrarily small bets.
  \end{itemize}
\end{frame}

\section{Go Razzmatazz, Go!}
\subsection{}
\begin{frame}{The Rules}
  \begin{enumerate}
  \item $n$ horses running, each with probability of winning $p_i$ and payout odds $r_i$
    \begin{enumerate}
      \item House advantage: $\sum_i \frac{1}{r_i} > 1$
    \end{enumerate}
  \item $\alpha_i$ units on each horse
  \item The horses run.
  \item Get paid according to your wager on the winning horse.
  \end{enumerate}

  \[ X = \sum_i \left( \alpha_i (-1 + r_i\mathbb{I}(h_i)) \right) \]
\end{frame}

\begin{frame}{Open Decisions}
  \begin{itemize}
  \item Which horse(s) to back.
  \item How much to bet.
  \item \sout{When to walk away.}
    \begin{itemize}
      \item We're going to be doing single round betting here.
    \end{itemize}
  \end{itemize}
\end{frame}

\begin{frame}{Pick the Right Horse}
  \begin{itemize}
  \item We could play it straight, as we did in roulette.
  \item More (any) epistemic uncertainty
  \item Might even be able to turn a profit, if you're smarter than the market.
  \item Still gambling though. We're here to win, \emph{guaranteed}.
  \end{itemize}
\end{frame}


\subsection{Some Low-Level Accounting}
\begin{frame}{The Booky's Favorite}
  \begin{itemize}
  \item Suppose there is an old nag in the race, Rocinante.
  \item Rocinante is a long, long, \emph{long} shot. Maybe 1 in 1000.
  \item Book-maker offers a special on Rocinante bets: 2000 to 1.
    \begin{itemize}
    \item Does not update the rest of the book.
    \item House advantage slips. Now $\sum_i \frac{1}{r_i} < 1$
    \end{itemize}
  \end{itemize}
\end{frame}

\begin{frame}{The Obvious Approach}
  \begin{itemize}
  \item Could just take the bookie up on the special.
  \item Bet is positive valued in expectation
  \item But I still need to make the rent this month, and this is a single round of gambling
  \item Too rich for my blood, can we do better?
  \end{itemize}
\end{frame}

\begin{frame}{Some Tulip Math}
  \[ X = \sum_i \left( \alpha_i (-1 + r_i\mathbb{I}(h_i)) \right) \]

  What happens if we try to clear $r_i$ out of our RV? Must have $\alpha_i = \frac{1}{r_i}$:

  \[ X = \sum_i \left( -\frac{1}{r_i} + \mathbb{I}(h_i)) \right) = 1 - \frac{1}{r_i}\]

  \pause

  That's a fixed payout.

  \pause

  A fixed, \emph{positive} payout. This is what's known as a Dutch Book.
\end{frame}

\begin{frame}{Alas!}
  Finding a bookie setting a line that is susceptible to a Dutch Book is\ldots
  not easy.
\end{frame}

\subsection{A Bigger, Dutcher Book}
\begin{frame}{$N$ bookies, One Race}
  \begin{itemize}
  \item Suppose multiple bookies are taking action on a single race.
  \item They have different clientelle, and are setting different lines
    \begin{itemize}
    \item Bookie doesn't care about odds, bookie wants to take a vig off of evenly spread bets.
    \item Just like we did with our Dutch book.
    \end{itemize}
  \item Call the most favorable odds across all book makers for each horse $r^*_i$
  \end{itemize}

  If $ \sum_i r_i^* < 1 $ we have a composite Dutch book! Let the arbitrage
times roll.
\end{frame}

\begin{frame}{Alas!}

Efficient market theory, like all theories, is wrong. But it's not often
wrong enough for this to be viable.

\pause

With the right combination of speed and luck, however, arbitrage is possible. A
frightening amount of effort goes into exactly these games, but played against
stock markets.

\pause

Also, we haven't so much beaten the house as become it.

\end{frame}

\end{document}
